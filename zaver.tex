\chapter*{Závěr}
\addcontentsline{toc}{chapter}{Závěr}
Produktem mé ročníkové práce je sbírka planimetrických a stereometrických úloh, která slouží k přípravě studentů na jednotné přijímací zkoušky z matematiky pro osmileté studium. Tuto sbírku mohou využít jak studenti, tak pedagogové, kteří chtějí své studenty na zkoušku připravit.

Vzhledem k tomu, že jsem se snažil co nejlépe napodobit styl úloh, které se objevují v jednotné zkoušce, věřím, že sbírka představuje efektivní nástroj pro úspěšné složení této zkoušky.

Hlavní předností sbírky je její zaměření na jednotné přijímací zkoušky, což pomůže studentům připravit se na planimetrické a stereometrické úlohy které se ve zkouškách objevují. Vzhled sbírky je navržen tak, aby byl co nejvíce přehledný a snadno srozumitelný.

Během tvorby sbírky jsem se naučil používat program GeoGebra pro tvorbu úloh a \LaTeX\ pro psaní práce. Největší výzvou byla tvorba samotných úloh, tato část mě ale zároveň nejvíce bavila.

Doufám, že má sbírka pomůže studentům připravit se na zkoušky a pomůže pedagogům při výuce matematiky.