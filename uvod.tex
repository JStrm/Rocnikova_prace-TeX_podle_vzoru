\chapter*{Úvod}
\addcontentsline{toc}{chapter}{Úvod}

Tato ročníková práce se zaměřuje na tvorbu sbírky planimetrických a stereometrických příkladů, která má pomoci studentům připravujícím se na střední školy a gymnázia. Výběr téma je motivován mým zájmem o tuto oblast matematiky a mými zkušenostmi s tím, jak někteří studenti mohou s těmito druhy úloh bojovat.

Práce se dělí na teoretickou část, vysvětlující jednotné přijímací zkoušky a teoretický základ, na praktickou část, ve které je popsán postup tvorby sbírky a na samotnou sbírku úloh.

Cílem sbírky je poskytnout přehledný materiál, který studentům umožní osvojit si potřebné dovednosti k řešení planimetrických a stereometrických úloh.

Sbírka bude strukturována dle témat jednotlivých úloh a bude přehledně čitelná a vizuálně přitažlivá.

Mým hlavním zdrojem jsou jednotné přijímací zkoušky z matematiky z minulých let. Také jsem se inspiroval různými sbírkami.

V rámci této práce je předpokládáno, že studenti již mají osvojenou znalost planimetrických a stereometrických konceptů, kterou získali v rámci školní výuky. Sbírka bude tedy obsahovat především různorodé příklady, které studentům umožní procvičit a zlepšit si své schopnosti v dané oblasti.
