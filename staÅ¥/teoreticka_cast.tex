\chapter{Teoretická část}

\section{Jednotné přijímací zkoušky pro střední školy a gymnázia}

Zkoušky tvoří příspěvková organizace CERMAT, neboli Centrum pro zjišťování výsledků vzdělávání, která byla zřízena ministerstvem školství, mládeže a tělovýchovy v roce 2006.~\cite{zakon_CERMAT} Tato organizace také zařizuje státní maturitní zkoušky a závěrečné zkoušky.~\cite{CERMAT_p_m}



Jde o národně jednotné přijímací zkoušky, které jsou povinnou součástí prvního kola přijímacího řízení do všech maturitních oborů s výjimkou oborů s talentovou zkouškou a oborů zkráceného studia.
Jednotná přijímací zkouška se skládá ze dvou písemných testů: z českého jazyka a literatury a z matematiky.
Varianty testů jsou různé pro čtyřleté obory vzdělání (včetně oborů nástavbového studia), pro šestiletá gymnázia a pro osmiletá gymnázia.
Maximální možný počet dosažených bodů v testech z matematiky i českého jazyka a literatury je 50 bodů.~\cite{CERMAT_co_to_je}

V České republice byly jednotné přijímací zkoušky testovány v letech 2015 a 2016, povinně zavedeny byly v roce 2017.~\cite{CERMAT_rocni_zprava}

Přijímací zkoušky z předchozích roků jsou dostupné na webových stránkách Centra pro zjišťování výsledků vzdělávání.~\cite{CERMAT_pdfka}


