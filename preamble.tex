%% Verze pro jednostranný tisk:
% Okraje: levý 40mm, pravý 25mm, horní a dolní 25mm
% (ale pozor, LaTeX si sám přidává 1in)

\setlength\textwidth{145mm}
\setlength\textheight{247mm}
\setlength\oddsidemargin{15mm}
\setlength\evensidemargin{15mm}
\setlength\topmargin{0mm}
\setlength\headsep{0mm}
\setlength\headheight{0mm}
% \openright zařídí, aby následující text začínal na pravé straně knihy
\let\openright=\clearpage

%% Pokud tiskneme oboustranně:
% \documentclass[12pt,a4paper,twoside,openright]{report}
% \setlength\textwidth{145mm}
% \setlength\textheight{247mm}
% \setlength\oddsidemargin{14.2mm}
% \setlength\evensidemargin{0mm}
% \setlength\topmargin{0mm}
% \setlength\headsep{0mm}
% \setlength\headheight{0mm}
% \let\openright=\cleardoublepage

%% Vytváříme PDF/A-2u
\usepackage[a-2u]{pdfx}

%% Přepneme na českou sazbu a fonty Latin Modern
\usepackage[czech]{babel}
\usepackage{lmodern}
\usepackage[T1]{fontenc}
\usepackage{textcomp}

%% Použité kódování znaků: obvykle latin2, cp1250 nebo utf8:
\usepackage[utf8]{inputenc}
\usepackage{csquotes}

%%% Další užitečné balíčky (jsou součástí běžných distribucí LaTeXu)
\usepackage{amsmath}        % rozšíření pro sazbu matematiky
\usepackage{amsfonts}       % matematické fonty
\usepackage{amsthm}         % sazba vět, definic apod.
\usepackage{amssymb}
\usepackage{bbding}         % balíček s nejrůznějšími symboly
% (čtverečky, hvězdičky, tužtičky, nůžtičky, ...)
\usepackage{bm}             % tučné symboly (příkaz \bm)
\usepackage{graphicx}       % vkládání obrázků
\graphicspath{ {img} }
\usepackage{fancyvrb}       % vylepšené prostředí pro strojové písmo
\usepackage{indentfirst}    % zavede odsazení 1. odstavce kapitoly     

\usepackage[backend=biber,style=numeric,sorting=none,block=ragged]{biblatex}
\addbibresource{zdroje.bib}

\usepackage{icomma}         % inteligetní čárka v matematickém módu
\usepackage{dcolumn}        % lepší zarovnání sloupců v tabulkách
\usepackage{booktabs}       % lepší vodorovné linky v tabulkách
\usepackage{paralist}       % lepší enumerate a itemize
\usepackage{xcolor}         % barevná sazba

\usepackage{setspace}        % řádkování
\usepackage{placeins}

%% Fix pro cmidrule and cline kvůli češtině
\usepackage{regexpatch}
\makeatletter
% Change the `-` delimiter to an active character
\xpatchparametertext\@@@cmidrule{-}{\cA-}{}{}
\xpatchparametertext\@cline{-}{\cA-}{}{}
\makeatother


%%% Údaje o práci
\title{
        {\Huge Tvorba sbírky planimetrických a~stereometrických úloh}\\
    {\Large Ročníková práce}\\
    {\includegraphics[width=0.24\textwidth]{mensa_logo_s_nazvem_vektor}}\\
    {\normalsize Mensa gymnázium, o.p.s.}
}
\author{
        {\Large Jan Strmiska}
}
\date{
    2021 - 2023
}

%% Balíček hyperref, kterým jdou vyrábět klikací odkazy v PDF,
%% ale hlavně ho používáme k uložení metadat do PDF (včetně obsahu).
%% Většinu nastavítek přednastaví balíček pdfx.
\hypersetup{unicode}
\hypersetup{breaklinks=true}

%% Znak obdelník
\newcommand{\rectangle}{\ooalign{$\sqsubset\mkern2mu$\cr$\mkern9mu\sqsupset$\cr}}

%% Nadpis kapitoly fix
\makeatletter
\def\@makechapterhead#1{
        {\parindent \z@ \raggedright \normalfont
    \Huge\bfseries \thechapter. #1
    \par\nobreak
    \vskip 20\p@
    }}
\def\@makeschapterhead#1{
        {\parindent \z@ \raggedright \normalfont
    \Huge\bfseries #1
    \par\nobreak
    \vskip 20\p@
    }}
\makeatother


\renewcommand{\angle}{\measuredangle}